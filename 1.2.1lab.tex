\documentclass[a4paper,10pt]{article} 
% --- Пакеты и кодировки ---
\usepackage[left=3cm,right=3cm]{geometry}   % поля
\usepackage{float}                           % [H] для рисунков

\usepackage[utf8]{inputenc}                  % кириллица в исходнике
\usepackage[T2A]{fontenc}                    % шрифтовая кодировка
\usepackage[russian]{babel}                  % локализация/переносы
\usepackage{csquotes}

% --- Графика (ВАЖНО: без dvips!) ---
\usepackage{graphicx}                        % Должен идти до \graphicspath
\graphicspath{{./}{pictures/}}               % ищем картинки и в текущей папке, и в pictures/
\DeclareGraphicsExtensions{.pdf,.png,.jpg}

% --- Математика ---
\usepackage{amsmath,amsfonts,amssymb,amsthm,mathtools}

% --- Ссылки/закладки ---
\usepackage{hyperref}                        % обычно подключают перед bookmark
\usepackage{bookmark}

% --- Границы плавающих объектов ---
\usepackage[section]{placeins}               % для \FloatBarrier

% --- Титул ---
\title{Отчёт о выполнении лабораторной работы\\
\textbf{Определение скорости полёта пули методом баллистического маятника}\\
\textbf{Лабораторная работа 1.2.1}}
\author{Г.А.Кузин }
\date{ Группа Б01-501 г. Долгопрудный, ФРКТ МФТИ, 30.11.2025 \\}

\begin{document}
\maketitle
\newpage

\section*{Аннотация}

Экспериментально определена скорость полёта пули из пневматического ружья методом баллистического маятника. Исследованы две модификации установки: маятник, совершающий поступательное движение тяжёлого цилиндра, и крутильный баллистический маятник.

На основе измерений отклонения маятника и периодов его колебаний по формулам, полученным из законов сохранения импульса и энергии~\cite{Landau,Saveliev}, вычислены значения скорости пули в каждом выстреле. Для поступательного маятника получено среднее значение
\[
\bar u_1 = (113{,}6 \pm 6{,}4)\ \text{м/с},
\]
для крутильного маятника
\[
\bar u_2 = (111{,}5 \pm 8{,}9)\ \text{м/с}.
\]
Результаты двух методов согласуются в пределах погрешностей, что подтверждает корректность применённой методики~\cite{Gladun}. Полученная скорость соответствует типичным значениям для учебных пневматических винтовок и может использоваться при контроле их энергетики.

\newpage
\section*{Введение}

В реальных задачах баллистики важную роль играет точное знание начальной скорости пули. Например, при сертификации пневматического спортивного оружия для тиров начальная энергия пули не должна превышать установленного регламентом значения. Для стандартной пули калибра $4{,}5$~мм это соответствует диапазону скоростей порядка $100$–$130$~м/с. 

На практике такие скорости часто измеряют с помощью оптических или инерционных хронографов. Однако в учебных лабораториях и при наладке стендов удобно использовать более простые механические методы, основанные на фундаментальных законах механики твёрдого тела~\cite{Landau}. Один из таких методов — баллистический маятник, позволяющий по отклонению маятника после удара определить скорость снаряда без прямого измерения времени его пролёта.


\newpage
\section*{Методика эксперимента}

В методике используются общие соотношения механики твёрдого тела и малых колебаний~\cite{Landau,Saveliev}. 

\subsection*{1. Поступательный баллистический маятник}

Рассматривается тяжёлый цилиндрический маятник массой $M$, подвешенный на длинных нитях длины $L$ так, что при малых отклонениях его центр масс движется по дуге окружности радиуса $L$. Пуля массы $m$ попадает в цилиндр и застревает в нём. Считается, что внешние горизонтальные силы за время удара малы, поэтому выполняется закон сохранения импульса системы «пуля + маятник»:
\begin{equation}
m u = (M + m) V,
\end{equation}
где $u$ — скорость пули перед ударом, $V$ — скорость маятника сразу после удара. Поскольку $M \gg m$, можно положить $M + m \approx M$ и записать
\begin{equation}
u \approx \frac{M}{m} V.
\label{eq:momentum}
\end{equation}

После удара маятник совершает колебания в поле тяжести. Пренебрегая потерями на трение и сопротивление воздуха, начальная кинетическая энергия маятника переходит в потенциальную энергию подъёма его центра масс, значит скорость равна
\begin{equation}
V^2 = 2 g h.
\label{eq:V2}
\end{equation}

Высота подъёма $h$ выражается через угол максимального отклонения $\varphi$ как
\begin{equation}
h = L(1-\cos\varphi) = 2L\sin^2\frac{\varphi}{2}.
\label{eq:h_general}
\end{equation}
При малых углах $\varphi$ имеем $\sin(\varphi/2) \approx \varphi/2$. Кроме того, горизонтальное смещение маятника $\Delta x$ связано с углом отклонения соотношением $\varphi \approx \Delta x/L$. 
\begin{equation}
h \approx \frac{\Delta x^2}{2L}.
\label{eq:h_small}
\end{equation}


Cкорость пули до удара выражается по формуле
\begin{equation}
u_1 = \frac{M}{m} \Delta x \sqrt{\frac{g}{L}}.
\label{eq:u1}
\end{equation}
\subsection*{2. Крутильный баллистический маятник}

Во второй части работы используется крутильный маятник. Пуля массой $m$ попадает в мишень, закреплённую на лёгком стержне на расстоянии $r$ от оси вертикальной проволоки. Считается, что за время удара внешние моменты малозначимы, и выполняется закон сохранения момента импульса
\begin{equation}
m u r = I \Omega,
\label{eq:Lz}
\end{equation}
где $I$ — момент инерции маятника относительно оси проволоки, $\Omega$ — угловая скорость сразу после удара.

Максимальный угол закручивания маятника $\varphi$ определяется из закона сохранения энергии при малых колебаниях и равен
\begin{equation}
\frac{k\varphi^2}{2} = \frac{I\Omega^2}{2},
\label{eq:energy}
\end{equation}
где $k$ — модуль кручения проволоки. Из~(\ref{eq:Lz}) и~(\ref{eq:energy}) следует
\begin{equation}
u_2 = \varphi \frac{\sqrt{kI}}{m r}.
\label{eq:u2_raw}
\end{equation}

Угол $\varphi$ измеряется оптически по смещению $x$ изображения нити на шкале, расположенной на расстоянии $d$ от оси вращения. При малых углах равен
\begin{equation}
\varphi \approx \frac{x}{2d}.
\label{eq:phi}
\end{equation}
\begin{equation}
u_2 = \frac{x}{2d}\,\frac{\sqrt{kI}}{m r}.
\label{eq:u2}
\end{equation}

Произведение $\sqrt{kI}$ определяется из измерений периода крутильных колебаний маятника с дополнительными грузами и без них. Пусть $T_1$ — период колебаний системы с грузами массой $M$, расположенными на расстоянии $R$ от оси, а $T_2$ — период той же системы без грузов. Тогда
\begin{equation}
T_1 = 2\pi\sqrt{\frac{I}{k}}, \qquad
T_2 = 2\pi\sqrt{\frac{I-2MR^2}{k}}.
\end{equation}
\section*{Результаты и их обсуждение}


Массы десяти пуль были измерены на весах. Результаты приведены в табл.~\ref{tab:masses}. Средняя масса пули составила
\[
\bar m = 0{,}505\ \text{г} \approx 5{,}05\cdot10^{-4}\ \text{кг}.
\]

Для поступательного маятника измерены длина подвеса и масса цилиндра:
\[
L = (2200 \pm 1)\ \text{мм}, \qquad
M_1 = (2925 \pm 5)\ \text{г}.
\]
Предварительное наблюдение показало, что за десять колебаний амплитуда уменьшается менее чем вдвое, поэтому затуханием можно пренебречь.

Во второй части работы использовался крутильный маятник с двумя симметричными грузами:
\[
R = (335 \pm 1)\ \text{мм},\quad
r = (220 \pm 1)\ \text{мм},\quad
d = (500 \pm 1)\ \text{мм},\quad
M_2 = (729{,}5 \pm 5)\ \text{г}.
\]
По измеренным периодам колебаний $T_1 = 15{,}3\ \text{с}$ и $T_2 = 17{,}4\ \text{с}$ с помощью формулы~(\ref{eq:kI}) найдено
\[
\sqrt{kI} = 64{,}8\cdot10^{-2}\ \text{кг}\cdot\text{м}^2/\text{с}.
\]

\subsection*{1. Поступательный баллистический маятник}

С помощью оптической системы для пяти выстрелов были измерены максимальные смещения маятника $\Delta x$, а затем по формуле~(\ref{eq:u1}) вычислены скорости пули $u_1$ (табл.~\ref{tab:exp1}). Среднее значение скорости
\[
\bar u_1 = 113{,}6\ \text{м/с},
\]
стандартное отклонение по серии выстрелов составляет $\sim 3{,}7$~м/с. Систематическая погрешность, рассчитанная по формуле~(\ref{eq:sigma_u1}), равна
\[
\sigma_{u_1} \approx 6{,}4\ \text{м/с}.
\]
Итоговый результат для первой установки:
\[
\bar u_1 = (113{,}6 \pm 6{,}4)\ \text{м/с}.
\]

Разброс отдельных результатов (столбец $u - \bar u_1$ в табл.~\ref{tab:exp1}) не превышает $5$~м/с и сопоставим как с расчётной погрешностью измерения, так и с естественной нестабильностью скорости выстрела пневматического оружия.

\subsection*{2. Крутильный баллистический маятник}

Во втором опыте для тех же пяти выстрелов измерялись смещения изображения нити $x$ на шкале. По формуле~(\ref{eq:u2}) были найдены скорости $u_2$ (табл.~\ref{tab:exp2}). Среднее значение скорости
\[
\bar u_2 = 111{,}5\ \text{м/с},
\]
а систематическая погрешность, рассчитанная по формуле~(\ref{eq:sigma_u2}), составляет
\[
\sigma_{u_2} \approx 8{,}9\ \text{м/с}.
\]
Таким образом,
\[
\bar u_2 = (111{,}5 \pm 8{,}9)\ \text{м/с}.
\]

Относительная погрешность во втором методе несколько больше, чем в первом, что объясняется накоплением ошибок при определении величины $\sqrt{kI}$ из формулы~(\ref{eq:kI}), а также дополнительными измерениями расстояний $r$ и $d$. При этом полученные значения скорости хорошо согласуются с результатами поступательного маятника:
\[
|\bar u_1 - \bar u_2| \approx 2\ \text{м/с} \ll \sigma_{u_1} + \sigma_{u_2},
\]
поэтому расхождение не является статистически значимым.

В пределах совместных погрешностей скорость пули можно считать равной
\[
u \approx (1{,}12 \pm 0{,}07)\cdot 10^{2}\ \text{м/с},
\]
что соответствует типичным паспортным данным учебных пневматических винтовок и подтверждает корректность работы обеих установок.

\newpage
\section*{Вывод}

В ходе работы экспериментально исследован метод определения скорости полёта пули с использованием поступательного и крутильного баллистических маятников. На основе законов сохранения импульса и механической энергии получены рабочие соотношения, позволяющие по измеренным смещениям маятника и периодам его колебаний вычислять скорость пули и оценивать погрешности результата.  

При обработке экспериментальных данных получены близкие значения скорости пули, определённые двумя независимыми методами; расхождение между ними не превышает суммарной экспериментальной погрешности, что свидетельствует о корректности сделанных допущений и работоспособности установки.  

Полученные результаты демонстрируют возможность надёжного определения начальной скорости пули в диапазоне, характерном для учебного пневматического оружия, и подтверждают пригодность рассматриваемой методики для практического контроля и калибровки пневматических винтовок, как отмечалось во введении.


\newpage
\section*{Приложение}
\appendix
\label{app:table}

В приложении приведены схемы установок и таблицы экспериментальных данных. На их основе по формулам~(\ref{eq:u1}) и~(\ref{eq:u2}) были рассчитаны скорости пули и оценены погрешности.

\subsection*{Схемы установок}

На рис.~\ref{fig:setup1} показана схема установки с поступательным баллистическим маятником. Цилиндрический маятник массы $M$ подвешен на нитях длины $L$ и может отклоняться в вертикальной плоскости. Пуля врезается в цилиндр, после чего по смещению изображения шкалы на экране определяется максимальное горизонтальное перемещение $\Delta x$.

\begin{figure}[h!]
\centering
\includegraphics[width=0.75\textwidth]{ballistic_pendulum_translational.png}
\caption{Схема установки для измерения скорости полёта пули с поступательным баллистическим маятником.}
\label{fig:setup1}
\end{figure}

На рис.~\ref{fig:setup2} представлена схема крутильного баллистического маятника. Пуля попадает в мишень на стержне на расстоянии $r$ от оси кручения проволоки. На том же стержне симметрично расположены грузы массы $M$ на расстоянии $R$, используемые для определения произведения $\sqrt{kI}$ по периоду крутильных колебаний. Отклонение маятника после выстрела фиксируется по смещению $x$ изображения нити на шкале, находящейся на расстоянии $d$ от оси.

\begin{figure}[h!]
\centering
\includegraphics[width=0.75\textwidth]{ballistic_pendulum_torsional.png}
\caption{Схема установки для измерения скорости полёта пули с крутильным баллистическим маятником.}
\label{fig:setup2}
\end{figure}

\FloatBarrier

\subsection*{ Таблицы экспериментальных данных}

Таблица~\ref{tab:masses} содержит значения масс десяти пуль, использованных в обеих сериях опытов. Среднее значение $\bar m$ использовалось при расчёте скоростей по формулам~(\ref{eq:u1}) и~(\ref{eq:u2}).

\begin{table}[h!]
\centering
\caption{Массы пуль}
\label{tab:masses}
\begin{tabular}{|c|c|c|c|c|c|c|c|c|c|c|}
\hline
№ пули & 1 & 2 & 3 & 4 & 5 & 6 & 7 & 8 & 9 & 10 \\
\hline
$m$, г & 0{,}500 & 0{,}514 & 0{,}500 & 0{,}518 & 0{,}500 & 0{,}505 & 0{,}503 & 0{,}499 & 0{,}500 & 0{,}518\\
\hline
\end{tabular}
\end{table}

В табл.~\ref{tab:exp1} приведены результаты измерения смещения $\Delta x$ поступательного маятника, соответствующие им скорости пули $u_1$, вычисленные по формуле~(\ref{eq:u1}), и отклонения отдельных значений от среднего $\bar u_1$.

\begin{table}[h!]
\centering
\caption{Результаты измерения скорости пули в опытах с установкой 1}
\label{tab:exp1}
\begin{tabular}{|c|c|c|c|}
\hline
$\Delta x$, мм & $u_1$, м/с & $u_1 - \bar u_1$, м/с \\
\hline
9{,}0 & 112{,}25 & -1{,}37 \\
9{,}5 & 115{,}26 & 1{,}67 \\
9{,}5 & 118{,}49 & 4{,}90 \\
9{,}0 & 108{,}35 & -5{,}24 \\
9{,}0 & 112{,}25 & -1{,}37 \\
\hline
\end{tabular}
\end{table}

Аналогично, в табл.~\ref{tab:exp2} приведены результаты для крутильного маятника: смещения $x$ изображения нити, скорости $u_2$, рассчитанные по формуле~(\ref{eq:u2}), и отклонения от среднего значения $\bar u_2$.

\begin{table}[h!]
\centering
\caption{Результаты измерения скорости пули в опытах с установкой 2}
\label{tab:exp2}
\begin{tabular}{|c|c|c|c|c|}
\hline
$x$, мм & $u_2$, м/с & $u_2 - \bar u_2$, м/с \\
\hline
18{,}5 & 108{,}97 & -2{,}53 \\
20{,}5 & 119{,}56 & 8{,}05 \\
19{,}0 & 111{,}25 & -0{,}25 \\
18{,}0 & 106{,}24 & -5{,}27 \\
19{,}0 & 111{,}25 & -0{,}25 \\
\hline
\end{tabular}
\end{table}

Табличные данные напрямую использовались при расчёте средних значений скорости, дисперсий результатов и оценке погрешностей, приведённых в разделе «Результаты и их обсуждение».

Рассматривается тяжёлый цилиндрический маятник массой $M$, подвешенный на длинных нитях длины $L$ так, что при малых отклонениях его центр масс движется по дуге окружности радиуса $L$. Пуля массы $m$ попадает в цилиндр и застревает в нём. Считается, что внешние горизонтальные силы за время удара малы, поэтому выполняется закон сохранения импульса системы «пуля + маятник»:
\begin{equation}
m u = (M + m) V,
\end{equation}
где $u$ — скорость пули перед ударом, $V$ — скорость маятника сразу после удара. Поскольку $M \gg m$, можно положить $M + m \approx M$ и записать
\begin{equation}
u \approx \frac{M}{m} V.
\label{eq:momentum}
\end{equation}

После удара маятник совершает колебания в поле тяжести. Пренебрегая потерями на трение и сопротивление воздуха, начальная кинетическая энергия маятника переходит в потенциальную энергию подъёма его центра масс
\begin{equation}
\frac{1}{2}(M+m)V^2 = (M+m) g h,
\end{equation}
откуда
\begin{equation}
V^2 = 2 g h.
\label{eq:V2}
\end{equation}

Высота подъёма $h$ выражается через угол максимального отклонения $\varphi$ как
\begin{equation}
h = L(1-\cos\varphi) = 2L\sin^2\frac{\varphi}{2}.
\label{eq:h_general}
\end{equation}
При малых углах $\varphi$ имеем $\sin(\varphi/2) \approx \varphi/2$. Кроме того, горизонтальное смещение маятника $\Delta x$ связано с углом отклонения соотношением $\varphi \approx \Delta x/L$. 
\begin{equation}
h \approx \frac{\Delta x^2}{2L}.
\label{eq:h_small}
\end{equation}

Cкорость маятника сразу после удара равна
\begin{equation}
V = \Delta x \sqrt{\frac{g}{L}}.
\end{equation}
Cкорость пуливыражается по формуле
\begin{equation}
u_1 = \frac{M}{m} \Delta x \sqrt{\frac{g}{L}}.
\label{eq:u1}
\end{equation}

Относительная погрешность определения $u_1$ при независимых ошибках измерения величин оценивается как
\begin{equation}
\sigma_{u_1} = 
u_1\sqrt{
\left(\frac{\Delta M}{M}\right)^2 +
\left(\frac{\Delta m}{m}\right)^2 +
\frac{1}{4}\left(\frac{\Delta L}{L}\right)^2 +
\left(\frac{\Delta_{\Delta x}}{\Delta x}\right)^2
}.
\label{eq:sigma_u1}
\end{equation}
Здесь $\Delta M$, $\Delta m$, $\Delta L$ и $\Delta_{\Delta x}$ — абсолютные инструментальные погрешности измерений массы маятника, массы пули, длины подвеса и смещения соответственно~\cite{Gladun}.
Считается, что за время удара внешние моменты малозначимы, и выполняется закон сохранения момента импульса:
\begin{equation}
m u r = I \Omega,
\label{eq:Lz}
\end{equation}
где $I$ — момент инерции маятника относительно оси проволоки, $\Omega$ — угловая скорость сразу после удара.

Максимальный угол закручивания маятника $\varphi$ определяется из закона сохранения энергии при малых колебаниях:
\begin{equation}
\frac{k\varphi^2}{2} = \frac{I\Omega^2}{2},
\label{eq:energy}
\end{equation}
где $k$ — модуль кручения проволоки. Из~(\ref{eq:Lz}) и~(\ref{eq:energy}) следует
\begin{equation}
u_2 = \varphi \frac{\sqrt{kI}}{m r}.
\label{eq:u2_raw}
\end{equation}

Угол $\varphi$ измеряется оптически по смещению $x$ изображения нити на шкале, расположенной на расстоянии $d$ от оси вращения. При малых углах
\begin{equation}
\varphi \approx \frac{x}{2d}.
\label{eq:phi}
\end{equation}
\begin{equation}
u_2 = \frac{x}{2d}\,\frac{\sqrt{kI}}{m r}.
\label{eq:u2}
\end{equation}

Произведение $\sqrt{kI}$ определяется из измерений периода крутильных колебаний маятника с дополнительными грузами и без них.
\begin{equation} 
T_1 = 2\pi\sqrt{\frac{I}{k}}, \qquad
T_2 = 2\pi\sqrt{\frac{I-2MR^2}{k}}.
\end{equation}
\begin{equation}
\sqrt{kI} = 
\frac{4\pi M R^2 T_1}{T_1^2 - T_2^2}.
\label{eq:kI}
\end{equation}

Систематическая погрешность определения $u_2$ оценивается по формуле
\begin{equation}
\sigma_{u_2} = 
u_2\sqrt{
\left(\frac{\Delta m}{m}\right)^2 +
\left(\frac{\Delta r}{r}\right)^2 +
\left(\frac{\Delta d}{d}\right)^2 +
\left(\frac{\Delta_{\sqrt{kI}}}{\sqrt{kI}}\right)^2 +
\left(\frac{\Delta_x}{x}\right)^2
},
\label{eq:sigma_u2}
\end{equation}
где $\Delta_{\sqrt{kI}}$ определяется из~(\ref{eq:kI}) методом дифференцирования, а $\Delta r$, $\Delta d$, $\Delta_x$ — инструментальные погрешности измерения соответствующих величин.


\FloatBarrier

\begin{thebibliography}{99}

\bibitem{Landau}
Л.\,Д. Ландау, Е.\,М. Лифшиц, \textit{Механика}. — М.: Наука, 1988. — 216~с.

\bibitem{Saveliev}
И.\,В. Савельев, \textit{Курс общей физики. Том II. Электричество и колебания}. — М.: Наука, 1989. — 560~с.

\bibitem{Gladun}
Д.\,А. Гладун, \textit{Лабораторный практикум по общей физике}. — М.: Издательство МФТИ, 2012. — 312~с.

\end{thebibliography}

\end{document}
